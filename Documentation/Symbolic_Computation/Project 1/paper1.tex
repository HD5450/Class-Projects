%===============================================
% MA 522
% Assignment 1
%===============================================
%===============================================
% Set up
%===============================================

\documentclass{article}%
\usepackage{amsmath}
\usepackage{amsfonts}
\usepackage{amssymb}
\usepackage{fullpage}
\usepackage{graphicx}
\usepackage{enumitem}%
\setcounter{MaxMatrixCols}{30}
%TCIDATA{OutputFilter=latex2.dll}
%TCIDATA{Version=5.50.0.2960}
%TCIDATA{CSTFile=LaTeX article (bright).cst}
%TCIDATA{Created=Friday, August 21, 2015 08:52:27}
%TCIDATA{LastRevised=Monday, October 07, 2019 20:07:17}
%TCIDATA{<META NAME="GraphicsSave" CONTENT="32">}
%TCIDATA{<META NAME="SaveForMode" CONTENT="1">}
%TCIDATA{BibliographyScheme=Manual}
%TCIDATA{<META NAME="DocumentShell" CONTENT="Standard LaTeX\Blank - Standard LaTeX Article">}
%BeginMSIPreambleData
\providecommand{\U}[1]{\protect\rule{.1in}{.1in}}
%EndMSIPreambleData
\allowdisplaybreaks
\newtheorem{theorem}{Theorem}
\newtheorem{acknowledgement}[theorem]{Acknowledgement}
\newtheorem{algorithm}[theorem]{Algorithm}
\newtheorem{axiom}[theorem]{Axiom}
\newtheorem{case}[theorem]{Case}
\newtheorem{claim}[theorem]{Claim}
\newtheorem{conclusion}[theorem]{Conclusion}
\newtheorem{condition}[theorem]{Condition}
\newtheorem{conjecture}[theorem]{Conjecture}
\newtheorem{corollary}[theorem]{Corollary}
\newtheorem{criterion}[theorem]{Criterion}
\newtheorem{definition}[theorem]{Definition}
\newtheorem{example}[theorem]{Example}
\newtheorem{exercise}[theorem]{Exercise}
\newtheorem{lemma}[theorem]{Lemma}
\newtheorem{notation}[theorem]{Notation}
\newtheorem{problem}[theorem]{Problem}
\newtheorem{proposition}[theorem]{Proposition}
\newtheorem{remark}[theorem]{Remark}
\newtheorem{solution}[theorem]{Solution}
\newtheorem{summary}[theorem]{Summary}
\DeclareMathOperator{\Tr}{Tr}
\newenvironment{proof}[1][Proof]{\noindent\textbf{#1.} }{\ \rule{0.5em}{0.5em}}
\begin{document}

\title{Counting number of distinct real roots of Pham system\\(Assignment 1)}
\author{Mountain Chan / B1-2}
\maketitle

%===============================================
%Cover page
%===============================================


%===============================================
\section{Problem}

\begin{definition}
[Pham system]A system $f\in\mathbb{C}\left[  x_{1},\ldots,x_{n}\right]^{n}$
is called a Pham system if it has the following form:%
\begin{align*}
f_{1}  &  =x_{1}^{d_{1}}+\sum_{e_{1}+\cdots+e_{n}<d_{1}}a_{1,e}x_{1}^{e_{1}}\cdots x_{n}^{e_{n}}\\
&  \vdots\\
f_{n}  &  =x_{1}^{d_{n}}+\sum_{e_{1}+\cdots+e_{n}<d_{n}}a_{n,e}x_{1}^{e_{1}}\cdots x_{n}^{e_{n}}%
\end{align*}

\begin{remark}
The pham system has exactly $m=d_{1}\cdots d_{n}$ many complex roots (counting multiplicies). 
Thus we will study the number of real roots of a real pham system.
\end{remark}
\end{definition}

\begin{problem}
Devise an algorithm with the following specification:

\begin{description}[leftmargin=3em,style=nextline,itemsep=0.5em]

\item[In:] $f\in\mathbb{R}\left[  x_{1},\ldots,x_{n}\right]^{n}$, Pham system

\item[Out:] $N,$ the number of distinct real roots of $f$
\end{description}
\end{problem}

\section{Theory}

\begin{definition}[Discriminant matrix]
Let  $f\in\mathbb{C}\left[  x_{1},\ldots,x_{n}\right] ^{n}$ be a Pham system of degrees 
$d_{1},\ldots,d_{n}$. 
Let $m=d_{1}\cdots d_{n}$. 
Let $z_{1},\ldots,z_{m}\in\mathbb{C}^{n}$ be the complex roots of $f$.
The discriminant matrix $Q$ of $f$ is defined by
\[
Q=VV^{t}%
\]
where
\[
V=\left[
\begin{array}
[c]{ccc}%
\omega_{1}\left(  z_{1}\right)   & \cdots & \omega_{1}\left(  z_{m}\right)  \\
\vdots                           &        & \vdots\\
\omega_{m}\left(  z_{1}\right)   & \cdots & \omega_{m}\left(  z_{m}\right)
\end{array}
\right]
\]
where%
\[
\omega=\left\{  x_{1}^{e_{1}}\cdots x_{n}^{e_{n}}:e_{1}<d_{1},\ldots,e_{n}<d_{n}\right\}
\]

\end{definition}

\begin{theorem}[Multivariate Hermite]
\# of distinct real roots of $f=\sigma\left(  Q\right)$
\end{theorem}

\begin{proof}
\begin{equation}
\begin{aligned}
Q_{ij} &= \sum_{k=1}^{m} \omega_{i}(z_k) \omega_{j}(z_k) \\
Q_{ij} &= \sum_{p=1}^{s} \omega_{}(\alpha_{p}) \mu_p \omega_{j}(\alpha_{p}) + \sum_{q=1}^{t} v_q (\omega_{i}(\beta_{q}) \omega_{j}(\beta_{q})) + \sum_{q=1}^{t} v_q (\overline{\omega_{i}(\beta_{q})} \, \overline{\omega_{j}(\beta_{q})}) \\
Q_{ij} &= \sum_{p=1}^{s} \omega_{i}(\alpha_{p}) \mu_p \omega_{j}(\alpha_{p}) + \sum_{q=1}^{t} v_q(\omega_{i}(\beta_{q}) \omega_{j}(\beta_{q}) \overline{\omega_{i}(\beta_{q})} \, \overline{\omega_{j}(\beta_{q})}) \\
Q_{ij} &= \sum_{p=1}^{s} \omega_{i}(\alpha_{p}) \mu_p \omega_{j}(\alpha_{p}) + \sum_{q=1}^{t} v_q(\omega_{i}(\beta_{q}) \omega_{j}(\beta_{q}) \overline{\omega_{i}(\beta_{q}) \omega_{j}(\beta_{q})}) \\
Q_{ij} &= \sum_{p=1}^{s} \omega_{i}(\alpha_{p}) \mu_p \omega_{j}(\alpha_{p}) + \sum_{q=1}^{t} 2v_q((\text{Re}(\omega_{i}(\beta_{q})) \text{Re}(\omega_{j}(\beta_{q}) - \text{Im}(\omega_{i}(\beta_{q})) \text{Im}(\omega_{j}(\beta_{q}))) \\
Q_{ij} &= \sum_{p=1}^{s} \omega_{i}(\alpha_{p}) \mu_p \omega_{j}(\alpha_{p}) + \sum_{q=1}^{t}  \text{Re}(\omega_{i}(\beta_{q})) (2v_q) \text{Re}(\omega_{j}(\beta_{q}) + \sum_{q=1}^{t} \text{Im}(\omega_{i}(\beta_{q})) (-2v_q) \text{Im}(\omega_{j}(\beta_{q})
\end{aligned}
\end{equation}
$$
D=\left[
\begin{array}{ccccccccc}
   \mu_{1}&&  \\
    &\ddots&  \\
    &&\mu_{s}\\
    &&&2v_1 \\
    &&&&\ddots\\
    &&&&&2v_{t}\\
    &&&&&&-2v_{1}\\
    &&&&&&&\ddots\\
    &&&&&&&&-2v_{t}
\end{array}
\right]
$$
$$
T = \left[
\begin{array}{ccccccccc}
   \omega_{0}(\alpha_{1}) & \cdots & \omega_{0}(\alpha_{s}) & \text{Re}(\omega_{0}(\beta_{1})) & \cdots & \text{Re}(\omega_{0}(\beta_{t})) & \text{Im}(\omega_{0}(\beta_{1})) & \cdots & \text{Im}(\omega_{0}(\beta_{t}))\\
   \vdots & \ddots & \vdots & \vdots & \ddots & \vdots & \vdots & \ddots & \vdots \\
   \omega_{m}(\alpha_{1}) & \cdots & \omega_{m}(\alpha_{s}) & \text{Re}(\omega_{m}(\beta_{1})) & \cdots & \text{Re}(\omega_{m}(\beta_{t})) & \text{Im}(\omega_{m}(\beta_{1})) & \cdots & \text{Im}(\omega_{m}(\beta_{t}))
\end{array}
\right]
$$
$\sigma(D) = (s+t)-t = s = $number of distinct real roots 
\\ By Sly. Law of Inertia, $\sigma(Q) = \sigma(D)$, hence number of distinct real roots of $f = \sigma(D) = s$
\end{proof}

\begin{remark}
Hermite's theorem is \textquotedblleft useless\textquotedblright\ because

\begin{enumerate}
\item Computation of $Q$ requires computing the roots of $f$.

\item Computation of $\sigma\left(  Q\right)$ requires computing the roots
of the characteristic polynomial of $Q$.
\end{enumerate}

\noindent Hence, to make Hermite's theorem useful, we need to find ways to

\begin{enumerate}
\item Compute $Q$\ \textbf{without} computing the roots of $f$.

\item Compute $\sigma\left(  Q\right)$\textbf{without} 
computing the roots of the characteristic polynomial of $Q$.
\end{enumerate}

\noindent In the following, we will tackle the challenges one by one.
\end{remark}

\begin{definition}
[Mutipilcation matrix]Let $g\in\mathbb{C}\left[  x_{1},\ldots,x_{n}\right]  $.
Then the multiplication matrix $M_{g}$ for $g$ is defined by%
\[
g\left[
\begin{array}
[c]{c}%
\omega_{1}\\
\vdots\\
\omega_{m}%
\end{array}
\right]  \equiv_{f}M_{g}\left[
\begin{array}
[c]{c}%
\omega_{1}\\
\vdots\\
\omega_{m}%
\end{array}
\right]
\]
\end{definition}

\begin{theorem}
We have $Q_{ij}=trM_{\omega_{i}\omega_{j}}$.
\end{theorem}

\begin{proof}
\begin{enumerate}
    \item Let $f(z) = 0$
    \\ $g(z)\left[
\begin{array}
[c]{c}%
\omega_{1}(z)\\
\vdots\\
\omega_{m}(z)%
\end{array}
\right]  = M_{g}\left[
\begin{array}
[c]{c}%
\omega_{1}(z)\\
\vdots\\
\omega_{m}(z)%
\end{array}
\right]$
    \item 
        $\begin{aligned}
             Q_{ij} &= \sum_{k=1}^{m} \omega_{i}(z_k) \omega_{j}(z_k) \\
             Q_{ij} &= \sum_{k=1}^{m} (\omega_{i}\omega_{j})(z_k) \\
             Q_{ij} &= \sum_{k=1}^{m} g(z_k) \\
             Q_{ij} &= trM_{\omega_{i}\omega_{j}}
        \end{aligned}$
\end{enumerate}
\end{proof}

\begin{theorem}
We have%
\[
Q_{ij}=\sum_{\mu\nu}M_{i\mu\nu}M_{j\nu\mu}%
\]
where%
\[
\omega_{i}\omega_{j}\equiv_{f}\sum_{k}M_{ijk}\omega_{k}%
\]
\end{theorem}

\begin{proof}
\begin{equation}
\begin{aligned}
             Q_{ij} &= \sum_{\mu} (M_{\omega_{i} \omega_{j}})_{\mu \mu} \\
             &= \sum_{\mu} (M_{\omega_{i}} M_{\omega_{j}})_{\mu \mu}
\end{aligned}
\end{equation}
\end{proof}



\section{Algorithm}

\begin{algorithm}[DiscriminantMatrix]\ 

\begin{description}
\item[In:] $f\in\mathbb{R}\left[  x_{1},\ldots,x_{n}\right]^{n}$, Pham system

\item[Out:] $Q,$ the discriminant matrix of $f$
\end{description}
\begin{enumerate}
\item $\omega = $ Monomial basis of function $f$
\item $p = $ dimension of matrix
\item $Q_{ij} = $ trace of matrix M
\item $Q_{ij} = tr M_{\omega_i \omega_j}$
\item Construct $Q$ with dimension $m x m$ from $Q_{ij}$
\item Return $Q$
\end{enumerate}
\end{algorithm}

\begin{remark}
Recall that Descartes' theorem is exact when all the roots are real.
\end{remark}

\begin{algorithm}[Signature]\ 

\begin{description}
\item[In:] $M\in\mathbb{R}^{m\times m}$, symmetric

\item[Out:] $S=\sigma(M)$
\end{description}

\begin{enumerate}
\item $C=$ $\left\vert \lambda I-M\right\vert $, the characteristic polynomial of $Q$

\item $m_{+}=$ the sign variation count of the coefficients of $C$

\item $m_{0}=$ the least exponent in $C$

\item $m_{-}=m-m_{+}-m_{0}$

\item $S=m_{+}-m_{-}$

\item Return $S$
\end{enumerate}
\end{algorithm}

\begin{algorithm}[NumberDistinctRealRoots]\ 

\begin{description}
\item[In:] $f\in\mathbb{R}\left[  x_{1},\ldots,x_{n}\right]  ^{n}$, Pham system

\item[Out:] $N$, the number of distinct real roots of $f$
\end{description}

\begin{enumerate}
\item $Q=DiscriminantMatrix\left(  f\right)  $

\item $N=Signature\left(  Q\right)  $

\item Return $N$
\end{enumerate}
\end{algorithm}


\end{document}
